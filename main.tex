\documentclass{amsart}
\usepackage{ifxetex}
\ifxetex
  \usepackage{fontspec}
  \usepackage{xunicode}
  \usepackage{xltxtra}
  \usepackage{xecyr}
  \setmainfont[Mapping=tex-text,Ligatures=TeX]{CMU Serif}
  \usepackage{polyglossia}
  \setdefaultlanguage{russian}
\else
  \usepackage[utf8]{inputenc}
  \usepackage[T2A]{fontenc}
  \usepackage[english,russian]{babel}
  \usepackage{concrete}
\fi
\usepackage{amsthm,amsmath,amsfonts,amssymb}
\usepackage{fullpage}
\usepackage{eufrak}
\usepackage{listings}
\usepackage{color}
\usepackage{xcolor}
\usepackage{stmaryrd}

\newtheorem{problem}{Задача}

\begin{document}

  \definecolor{dkgreen}{rgb}{0,0.6,0}
  \definecolor{gray}{rgb}{0.5,0.5,0.5}
  \definecolor{mauve}{rgb}{0.58,0,0.82}  

  \newcommand{\problemset}[1]{
    
    \begin{center}
      \Large #1
    \end{center}
  }

  \lstset{ %
    language=C++,                % the language of the code
    basicstyle=\footnotesize,           % the size of the fonts that are used for the code
    numbers=left,                   % where to put the line-numbers
    numberstyle=\tiny\color{gray},  % the style that is used for the line-numbers
    stepnumber=1,                   % the step between two line-numbers. If it's 1, each line 
                                    % will be numbered
    numbersep=5pt,                  % how far the line-numbers are from the code
    backgroundcolor=\color{white},      % choose the background color. You must add \usepackage{color}
    showspaces=false,               % show spaces adding particular underscores
    showstringspaces=false,         % underline spaces within strings
    showtabs=true,                 % show tabs within strings adding particular underscores
    frame=single,                   % adds a frame around the code
    rulecolor=\color{black!10},        % if not set, the frame-color may be changed on line-breaks within not-black text (e.g. comments (green here))
    tabsize=2,                      % sets default tabsize to 2 spaces
    captionpos=b,                   % sets the caption-position to bottom
    breaklines=true,                % sets automatic line breaking
    breakatwhitespace=false,        % sets if automatic breaks should only happen at whitespace
    title=\lstname,                   % show the filename of files included with \lstinputlisting;
                                    % also try caption instead of title
    keywordstyle=\color{blue},          % keyword style
    commentstyle=\color{dkgreen},       % comment style
    stringstyle=\color{mauve},        % string literal style
    escapeinside={\%*}{*)},            % if you want to add LaTeX within your code
    morekeywords={done, to},              % if you want to add more keywords to the set
  %  deletekeywords={...}              % if you want to delete keywords from the given language
  }

  \begin{tabbing}
\hspace{11cm} \= Студент: \= Городилов Евгений \\
  \> Группа: \> 2304 \\
  \> Дата: \> \today
\end{tabbing}
\hrule
\vspace{1cm}
  \begin{problem}
  Выписать семантику выражений языка While
  
  \begin{equation}
    \llbracket n \rrbracket  = \lambda s.n
  \end{equation}

  \begin{equation}
      \llbracket x \in V  \rrbracket = \lambda s.s \ x
  \end{equation}
  
  \begin{equation}
      \llbracket A \ (+,\ *) \ B \rrbracket = 
        \lambda s. \llbracket A \rrbracket \ s \ (+,\ *) \ \llbracket B \rrbracket \ s
  \end{equation}

  \begin{equation}
    \llbracket A - B \rrbracket = \lambda s.
    \begin{cases}
      \llbracket A \rrbracket \ s - \llbracket B \rrbracket \ s, \ \llbracket A \rrbracket \ s > \llbracket B \rrbracket \ s \\
      undef,\ otherwise
    \end{cases}
  \end{equation}

  \begin{equation}
    \llbracket A \ (/,\ \%) \ B \rrbracket = \lambda s.
    \begin{cases}
      \llbracket A \rrbracket \ s \ (/,\ \%) \ \llbracket B \rrbracket \ s, \ \llbracket B \rrbracket \ s \ne 0 \\
      undef,\ otherwise
    \end{cases}
  \end{equation}

  \begin{equation}
    \llbracket A \ \otimes \ B \rrbracket = \lambda s.
    \begin{cases}
      1, \ \llbracket A \rrbracket \ s \ \otimes \  \llbracket B \rrbracket \ s \\
      0,\ otherwise
    \end{cases}
    \otimes = \{ ==,\ \ne,\ >,\ \geq, <,\ \leq \} 
  \end{equation}

  \begin{equation}
    \llbracket x \in \{1,\ 0 \} \ (\&\&,\ ||) \ y \in \{1,\ 0 \} \rrbracket = \lambda s.
      x \ (\&\&,\ ||) \ y \
  \end{equation}

\end{problem}

\newpage

\begin{problem}
  Вычислить семантику следующей программы на языке While
  
  \begin{lstlisting}
  read(n);
  i := 1;
  while n > 0 do {
    i := i * n;
    n := n - 1
  };
  write(i)
  \end{lstlisting}
\end{problem}

$(\bot,\ n,\ o) \xrightarrow{read(x)} (s\ [n \gets x],\ i,\ o)$ \\

$(s\ [n \gets x],\ i,\ o) \xrightarrow{i := 1} (s\ [n \gets x,\ i \gets 1],\ i,\ o)$ \\

\[(s\ [n \gets x,\ i \gets 1],\ i,\ o) \xrightarrow{i := i * n} (s\ [n \gets x,\ i \gets i*x],\ i,\ o)\]

\[(s\ [n \gets x,\ i \gets i*x],\ i,\ o) \xrightarrow{n := n - 1} (s\ [n \gets x - 1,\ i \gets i*x],\ i,\ o)\]

\[\uparrow\]

\[(s\ [n \gets x,\ i \gets 1],\ i,\ o) \xrightarrow{while} (s\ [\ n \gets 0,\ i \gets x!],\ i,\ o)\] \\

$(s\ [n \gets 0,\ i \gets x!],\ i,\ o) \xrightarrow{write(i)} (s\ [n \gets 0,\ i \gets x!],\ i,\ o:[x!])$

\end{document}
